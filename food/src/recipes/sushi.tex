\subsection{Sushi}
\qindex{Sushi} is a dish easy to prepare, but with a certain flair. Required tools are only a cooking pot (to boil the rice), a bamboo-mat (or another flexible mat of similar size) to roll the Mali, and a really sharp knife. For the right flair, however, a set of small Japanese bowls, chopsticks and some nice chinaware to serve the sushi.
\subsubsection{Ingredients}
\begin{minipage}[t]{0.5\textwidth}
{\bf Mandatory}
  \begin{itemize}
  \item{rice (1 cup)- either special sushi-rice or round-corned rice like Kome or Nishiki}
  \item{rice vinegar ($1/6^{th}$ cup) - can be substituted by ordinary vinegar with a bit sugar and salt}
  \item{sugar (1 teaspoon)}
  \item{salt ($1/2$ teaspoon)}
  \item{nori-leaves (the sheets of seaweed)}
  \item{soy sauce}
  \item{wasabi}
  \item{gari (sweet and sour pickled ginger)}
  \end{itemize}
\end{minipage}
\begin{minipage}[t]{0.5\textwidth}
{\bf Optional}
  \begin{itemize}
  \item{sesame}
  \item{cucumber}
  \item{bell pepper}
  \item{salmon}
  \item{suremi - crab-meat surrogate}
  \end{itemize}
\end{minipage}
\subsubsection{Preparation}
Cook the rice and let it cool down. Rice can be cooked a day ahead and kept in the fridge, or - if the decision to make sushi was not planned ahead - distributed on a baking tray to cool down faster.\\
Mix first the vinegar with sugar and salt until all crystals are resolved, then mix it with rice.\\
For {\it nigiri}\index{sushi, nigiri}, form small lumps of rice, put some small slice of fish, prawn, caviar or other seafood on top, and if you like, bind it with a narrow strip of nori to keep it together.\\
For {\it maki}\index{sushi, maki} you will need the bamboo-mat, and probably some plastic wrap to avoid some mess. Cover the mat with the plastic wrap. Put a nori leave on it, shiny side down. Cover the nori-sheet with 1cm of the rice. Roll it up and use a really sharp knife to slice it.
\vfill
